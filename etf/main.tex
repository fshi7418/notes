%%%%%%%%%%%%%%%%%%%%%%%%%%%%%%%%%%%%%%%%%
% The Legrand Orange Book
% LaTeX Template
% Version 2.4 (26/09/2018)
%
% This template was downloaded from:
% http://www.LaTeXTemplates.com
%
% Original author:
% Mathias Legrand (legrand.mathias@gmail.com) with modifications by:
% Vel (vel@latextemplates.com)
%
% License:
% CC BY-NC-SA 3.0 (http://creativecommons.org/licenses/by-nc-sa/3.0/)
%
% Compiling this template:
% This template uses biber for its bibliography and makeindex for its index.
% When you first open the template, compile it from the command line with the 
% commands below to make sure your LaTeX distribution is configured correctly:
%
% 1) pdflatex main
% 2) makeindex main.idx -s StyleInd.ist
% 3) biber main
% 4) pdflatex main x 2
%
% After this, when you wish to update the bibliography/index use the appropriate
% command above and make sure to compile with pdflatex several times 
% afterwards to propagate your changes to the document.
%
% This template also uses a number of packages which may need to be
% updated to the newest versions for the template to compile. It is strongly
% recommended you update your LaTeX distribution if you have any
% compilation errors.
%
% Important note:
% Chapter heading images should have a 2:1 width:height ratio,
% e.g. 920px width and 460px height.
%
%%%%%%%%%%%%%%%%%%%%%%%%%%%%%%%%%%%%%%%%%

%----------------------------------------------------------------------------------------
%	PACKAGES AND OTHER DOCUMENT CONFIGURATIONS
%----------------------------------------------------------------------------------------

\documentclass[11pt,fleqn]{book} % Default font size and left-justified equations

\input{structure.tex} % Insert the commands.tex file which contains the majority of the structure behind the template

%\hypersetup{pdftitle={Title},pdfauthor={Author}} % Uncomment and fill out to include PDF metadata for the author and title of the book

%----------------------------------------------------------------------------------------

\begin{document}

%----------------------------------------------------------------------------------------
%	TITLE PAGE
%----------------------------------------------------------------------------------------

\begingroup
\thispagestyle{empty} % Suppress headers and footers on the title page
\begin{tikzpicture}[remember picture,overlay]
\node[inner sep=0pt] (background) at (current page.center) {\includegraphics[width=\paperwidth]{background.pdf}};
\draw (current page.center) node [fill=ocre!30!white,fill opacity=0.6,text opacity=1,inner sep=1cm]{\Huge\centering\bfseries\sffamily\parbox[c][][t]{\paperwidth}{\centering Bundled Linear Combinations of Financial Instruments \\[15pt] % Book title
{\Large f'\{subtitle\}'\\[20pt]} % Subtitle
{\huge Frank Shi}
}}; % Author name
\end{tikzpicture}
\vfill
\endgroup

%----------------------------------------------------------------------------------------
%	COPYRIGHT PAGE
%----------------------------------------------------------------------------------------

\newpage
~\vfill
\thispagestyle{empty}

\noindent Copyright \copyright\ January 2023 Frank Shi\\ % Copyright notice

%\noindent \textsc{Not Published}\\ % Publisher

% \noindent \textsc{book-website.com}\\ % URL

%\noindent Licensed under the Creative Commons Attribution-NonCommercial 3.0 Unported License (the ``License''). You may not use this file except in compliance with the License. You may obtain a copy of the License at \url{http://creativecommons.org/licenses/by-nc/3.0}. Unless required by applicable law or agreed to in writing, software distributed under the License is distributed on an \textsc{``as is'' basis, without warranties or conditions of any kind}, either express or implied. See the License for the specific language governing permissions and limitations under the License.\\ % License information, replace this with your own license (if any)
%
%\noindent \textit{First printing, March 2019} % Printing/edition date

%----------------------------------------------------------------------------------------
%	TABLE OF CONTENTS
%----------------------------------------------------------------------------------------

%\usechapterimagefalse % If you don't want to include a chapter image, use this to toggle images off - it can be enabled later with \usechapterimagetrue

\chapterimage{tocontents.pdf} % Table of contents heading image

\pagestyle{empty} % Disable headers and footers for the following pages

\tableofcontents % Print the table of contents itself

\cleardoublepage % Forces the first chapter to start on an odd page so it's on the right side of the book

\pagestyle{fancy} % Enable headers and footers again

%----------------------------------------------------------------------------------------
%	PART
%----------------------------------------------------------------------------------------

%\part{Part One}

%----------------------------------------------------------------------------------------
%	CHAPTER 1
%----------------------------------------------------------------------------------------

\chapterimage{chapter_head_1.pdf} % Chapter heading image

\chapter{Fundamental Concepts}

\section{Baskets}

\begin{remark} \label{rmk:111}
Prices and definitions in general are static data, i.e. assume that we fix a point in time during all
subsequent discussions.\\
Moreover, the set of natural numbers \(\bN\) consists of integers greater or equal to 1, i.e.
\[
\bN = \{1, 2, \ldots \},
\]
and the set of positive real numbers is \(\bR^+ := \bR \setminus \{0\}\),
\end{remark}

\begin{definition} \label{def:112}
\index{Financial instrument}
\index{Base currency}
Fix a currency \(C_B\), a \textbf{financial instrument} \(I\) with respect to \(C_B\) is a 3-tuple
\[
(Q_t, P_t, F_t)
\]
where \(Q_t \in \bR\) is the quantity of the instrument in existence at time \(t\), \(P_t \in \bR^+\) is a random variable that represents the price of the instrument at time \(t\) and
\(F_t \in \bR^+\) is such that \(P_tF_t\) represents the price of the instrument in \(C_B\) at time \(t\).\\
The currency \(C_B\) is known as the \textbf{base currency}.
\end{definition}

\begin{definition} \label{def:113}
\index{Cash instrument}
A financial instrument \(I\) for which \(P_t = 1\) for all \(t\) is a \textbf{cash instrument}. Any financial
instrument that is not a cash instrument is a \textbf{non-cash instrument}.
\end{definition}

\begin{definition} \label{def:114}
\index{Market basket}
\index{Cash basket}
\index{Quantity vector}
Fix a currency \(C_B\), a \textbf{non-cash basket} or a \textbf{market basket} of size \(n\) is a finite set of non-cash instruments
\[
\{I_1 := (Q^1_t, P^1_t, F^1_t), \ldots, I_n := (Q_t^n, P_t^n, F_t^n)\}
\]
where \(Q_t^i \in \bR\) represents the amount of \(I_i\) held at time \(t\), 
\(P_t^i \in \bR^+\) represents the price of \(I_i\) at time \(t\), and \(F_t^i \in \bR^+\) is 
such that \(P_t^iF_t^i\) represents the price of \(I_i\) in currency \(C_B\) at time \(t\).\\
A \textbf{cash basket} of size \(m\) is a finite set of cash instruments
\[
\{C_1 := (q_t^1, 1, f_t^1), \ldots, C_m := (q_t^m, 1, f_t^m)\}.
\]
\end{definition}

\begin{remark} \label{rmk:115}
\index{Local currency}
Whenever a market instrument or a cash instrument is discussed, it is always with respect to a base currency 
\(C_B\). This is because the instruments in general might
be in many different currencies, commonly referred to as \textbf{local currencies}. Due to reasons beyond the 
scope of this note, given two local currencies \(C_1\) 
and \(C_2\), it is in general not true that 1 \(C_1\) can buy the same amount of things as 1 \(C_2\). Because of this, to compare the prices of two instruments in different
currencies fairly, we need to convert their local currencies into the same currency. That currency we convert to is the base currency \(C_B\).
\end{remark}

\begin{definition} \label{def:116}
\index{Bundle}
A \textbf{bundle} is a financial instrument with a representation of a 2-tuple
\[
(M, C)
\]
where \(M\) is a market basket and \(C\) is a cash basket such that \(M\) and \(C\) share the same base currency \(C_B\).
\end{definition}

\begin{proposition} \label{prop:117}
The cash value of a bundle \((M, C)\) at time \(t\) in base currency \(C_B\) is
\[
P_b(t) = P_M(t) + P_C(t)
\]
where \(P_M(t)\), the price of the market basket at time \(t\) in base currency \(C_B\), is given by
\[
P_M(t) = \sum_{i=1}^n Q_t^iP_t^iF_t^i,
\]
and \(P_C(t)\), the price of the cash basket at time \(t\) in base currency \(C_B\), is given by
\[
P_C(t) = \sum_{i=1}^m q_t^if_t^i.
\]
\end{proposition}
\begin{proof}
This is straightforward by the definition of the bundle and the baskets.
\end{proof}

%------------------------------------------------

\section{Subscription and Redemption}

\begin{definition} \label{def:121}
\index{Portfolio}
A \textbf{portfolio} is a finite set of financial instruments
\[
\{I_1, \ldots, I_n\}.
\]
\end{definition}

\begin{definition} \label{def:122}
\index{Subscription!In-kind}
\index{Subscription!Cash}
\index{Redemption!In-kind}
\index{Redemption!Cash}
Given a portfolio \(P := \{I_1, \ldots, I_n\}\), a bundle \((M, C)\), an \textbf{in-kind subscription} is defined as the operation
\[
P \cup \{(M, C)\} \setminus (M \cup C),
\]
and an \textbf{in-kind redemption} is defined as the operation
\[
P \setminus \{(M, C)\} \cup (M \cup C).
\]
\indent At a fixed time \(t\), suppose \(M\) has price \(P_M(t)\) and \(C\) has price \(P_C(t)\) in base currency
\(C_B\). Define a cash basket \(B := \{(1, 1)\}\) with 1-dimensional quantity vector \([P_M(t) + P_C(t)]\), then a \textbf{cash subscription} is the operation
\[
P \cup \{(M, C)\} \setminus \{B\}
\]
and a \textbf{cash redemption} is the operation
\[
P \cup \{B\} \setminus \{(M, C)\}.
\]
\end{definition}

\begin{remark} \label{rmk:123}
In words, an in-kind subscription (redemption) is to give away the baskets (bundle) and get back the bundle
(baskets). On the other hand, a cash subscription is to give away the cash value of the baskets and get back the
bundle, while a cash redemption is to give away the bundle and get back the cash value of the baskets. Both cash
subscription and redemption is done in the base currency \(C_B\).
\end{remark}



%------------------------------------------------

%----------------------------------------------------------------------------------------
%	CHAPTER 2
%----------------------------------------------------------------------------------------

\chapterimage{chapter_head_3.pdf} % Chapter heading image

\chapter{Pricing}

%------------------------------------------------

\section{Useful Values}

%------------------------------------------------

%----------------------------------------------------------------------------------------
%	CHAPTER 3
%----------------------------------------------------------------------------------------

\chapterimage{chapter_head_2.pdf} % Chapter heading image

\chapter{Hedging}

%------------------------------------------------

\section{Proxy}

%------------------------------------------------

\section{Fixed Income}

%------------------------------------------------
%	INDEX
%----------------------------------------------------------------------------------------

\cleardoublepage % Make sure the index starts on an odd (right side) page
\phantomsection
\setlength{\columnsep}{0.75cm} % Space between the 2 columns of the index
\addcontentsline{toc}{chapter}{\textcolor{ocre}{Index}} % Add an Index heading to the table of contents
\printindex % Output the index

%----------------------------------------------------------------------------------------

\end{document}
